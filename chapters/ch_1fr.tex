% 10/11/2020 - First round of corrections: added Thomas' corrections
\chapter*{Préface}\addcontentsline{toc}{chapter}{Préface}

L'amélioration de la qualité des sources de rayons X modernes impose des exigences de plus en plus strictes à la qualité de l'optique des rayons X, qui, lorsqu'elle est utilisée pour focaliser les faisceaux de rayons X, ne doit pas dégrader la brillance de la source de rayons X. Pour y parvenir, il faut réduire au minimum la perturbation du front d'onde des rayons X, les effets néfastes sur les points focaux et les pertes d'intensité. Pour y parvenir, les lentilles à rayons X doivent présenter une fidélité de forme, des surfaces lisses et une structure interne homogène et pure. 

Ce projet de doctorat visait à déterminer l'effet des erreurs de forme des lentilles réfractives, de la rugosité de surface et des impuretés sur un faisceau de rayons X partiellement cohérent ayant des caractéristiques similaires aux lignes de faisceaux de l'onduleur après la mise à niveau de l'ESRF-EBS. Sur la base de développements récents, l'atténuation des erreurs de forme des lentilles à l'aide d'optiques correctives a également été étudiée. Pour atteindre les objectifs proposés, ce projet reposait sur deux piliers : théorique et expérimental, avec des aspects techniques liés aux deux. Ce projet a abordé des aspects importants du programme de R\&D en optique des rayons X de l'ESRF-EBS, tel qu'il est défini dans le plan stratégique de mise à niveau (Orange book).

La facette théorique du travail a consisté à étudier les limites de la modélisation actuelle et les approximations utilisées dans la plupart des codes de simulation pour traiter les éléments optiques. Après avoir évalué l'adéquation des outils existants, la proposition d'extensions et de nouveaux développements. L'objectif était également d'ajouter aux simulations la capacité de traiter les données de métrologie et de développer un cadre pour la conception d'optiques réfractives correctives. Intégrer les modèles de simulation à des simulations cohérentes et partiellement cohérentes pour obtenir de manière réaliste l'effet des imperfections optiques sur un faisceau de rayons X et comparer les résultats avec la littérature et les données expérimentales. Les objectifs techniques liés à cette partie théorique comprenaient le développement de bibliothèques Python permettant d'utiliser facilement la modélisation nouvellement développée avec le code "Synchrotron Radiation Workshop" (SRW) pour la conception des lignes de faisceaux. Une partie de cette implémentation a été développée en collaboration avec O. Chubar (auteur du SRW) au cours de deux visites scientifiques au Brookhaven National Lab. aux États-Unis. Cette bibliothèque Python nouvellement développée a servi de base à une intégration plus poussée dans des interfaces graphiques utilisateur telles que OrAnge SYnchrotron Suit (OASYS). 

Afin d'obtenir des résultats réalistes pour les simulations, des techniques de détection du front d'onde en champ proche basées sur le speckle des rayons X ont été utilisées pour caractériser les lentilles produites en interne ; les lentilles et l'optique à forme libre dans le cadre de collaborations scientifiques ; et les lentilles commerciales nouvellement acquises. La métrologie des temps de propagation dans la longueur d'onde a été effectuée de manière systématique à la BM05 de l'ESRF avant son arrêt début décembre 2018, à la ligne de faisceaux 1-BM de l'APS à Chicago en 2019 et à la ligne de faisceaux ID06 pendant la période de mise en service de l'ESRF-EBS. Outre la mesure de l'optique des rayons X et la création d'une base de données de métrologie pour les lentilles à rayons X, les objectifs techniques de cette partie expérimentale étaient de former le candidat au doctorat pour qu'il soit capable de réaliser avec compréhension, compétence et autonomie : la mise en place du dispositif expérimental à la ligne de faisceaux, l'acquisition et le traitement des données. Le développement de protocoles de mesure internes en termes d'alignement des sondes et la standardisation des mesures et de l'analyse des données étaient également prévus. 

À un stade ultérieur, il fallait également étudier les récents développements des techniques de fabrication additive et soustractive pour la fabrication de la correction optique. Concevoir et tester la performance sur un faisceau de rayons X des correcteurs de première phase à l'ESRF était envisagé comme une façon naturelle de conclure la formation doctorale.

\section*{Aperçu}\addcontentsline{toc}{section}{Aperçu}

Ce travail est divisé en six chapitres, une conclusion et une annexe résumant les publications pertinentes de l'auteur au cours de ce projet de doctorat :
\\
\\
\textbf{Chapitre \ref{sec:x-rays}~-~\nameref{sec:x-rays}} est le premier de deux chapitres essentiellement théoriques. Le titre est un clin d'œil au titre de la publication qui relate la découverte des rayons X en 1895. Il commence par introduire les concepts de brillance et les relie aux sources de rayons X basées sur les accélérateurs. Les sources de rayons X à haute brillance sont ensuite présentées et le concept de rayons X latéralement cohérents est illustré. L'optique physique est ensuite présentée comme la description la plus appropriée des champs (partiellement) cohérents. La propagation en espace libre et l'approximation paraxiale sont expliquées et la propagation des rayons X à travers la matière est modélisée par l'introduction de l'élément de transmission. Une brève discussion sur la cohérence optique et la présentation de quelques concepts de base sont faites à la fin de cette section. Ce chapitre se termine par une discussion sur les simulations, méthodes et approches optiques des rayons X. Tout au long du texte, de nombreux ouvrages recommandés sont présentés.
\\
\\
\textbf{Chapitre \ref{sec:x-ray_optics}~-~\nameref{sec:x-ray_optics}}, une référence à la conférence de A. Compton sur le Nobel, s'ouvre sur un récit historique des débuts de la science des rayons X montrant les étapes qui ont conduit à la compréhension des rayons X comme branche de l'optique. Un bref examen des développements de l'optique de focalisation des rayons X, divisé par les phénomènes optiques, est donné pour contextualiser l'évolution récente de l'optique réfractive. La modélisation de la lentille idéale pour les rayons X et l'empilement idéal de lentilles basé sur des techniques de type multi-tranches sont présentés. Avec peu de modifications, ce modèle peut accepter des cartes d'erreurs arbitraires de figures 2D pour tenir compte des imperfections optiques. Des mesures importantes pour l'évaluation des performances du LCR sont introduites et les conditions de tolérance des aberrations sont présentées. Ce chapitre conclut la présentation des aspects théoriques nécessaires à cette thèse.
\\
\\
\textbf{Chapitre \ref{sec:modelling}~-~\nameref{sec:modelling}} se concentre sur la modélisation de la lentille de rayons X. En ajoutant aux surfaces de focalisation avant et arrière des degrés de liberté latéraux et angulaires, il est possible d'imiter les désalignements et les erreurs de fabrication typiques rencontrés dans les lentilles réelles. Dans les cas où les nouveaux degrés de liberté paramétrés ne suffisent pas, la modélisation d'erreurs de forme plus complexes est possible en utilisant les polynômes orthonormaux Zernike ou 2D Legendre ou des données de métrologie. Ce chapitre se termine par les détails des bibliothèques Python implémentées pour la modélisation des imperfections de phase dans les lentilles de rayons X.
\\
\\
\textbf{Chapitre \ref{sec:measuring}~-~\nameref{sec:measuring}} présente une description complète de la technique de suivi des vecteurs de speckle en champ proche par rayons X (XSVT) utilisée pour inspecter les lentilles utilisées dans cette thèse. Il commence par décrire la pluralité des techniques de métrologie en longueur d'onde et explique pourquoi la technique XSVT est la plus appropriée pour ce travail. Un examen des principaux aspects du dispositif expérimental, de l'acquisition, du traitement et de l'analyse des données est présenté et une discussion sur la métrologie des lentilles à rayons X par rapport aux piles de lentilles clôt ce chapitre.
\\
\\
\textbf{Chapitre \ref{sec:effect_optical_imperfections}~-~\nameref{sec:effect_optical_imperfections}} traite de l'effet des imperfections optiques sur la dégradation du faisceau de rayons X en présentant une vaste collection de simulations de la cause du faisceau, de la fonction d'étalement ponctuel et du profil du faisceau à des positions sélectionnées le long de l'axe optique pour plusieurs configurations optiques. Une discussion sur la métrologie des lentilles individuelles par rapport aux lentilles empilées du point de vue des simulations est présentée, suivie d'une discussion sur l'effet des imperfections optiques et l'adéquation du rapport de Strehl pour les lentilles à rayons X. Quelques commentaires sur le temps de simulation concluent le chapitre.
\\
\\
\textbf{Chapitre \ref{sec:corrections}~-~\nameref{sec:corrections}} est le dernier chapitre et conclut le voyage qui a commencé dans \textbf{Chapitre~\ref{sec:x-ray_optics}} avec la modélisation des lentilles idéales, en passant par la modélisation des lentilles aberrantes dans \textbf{Chapitre~\ref{sec:modelling}}, en mesurant les mêmes imperfections dans \textbf{Chapitre~\ref{sec:measuring}} et en comprenant leurs effets sur un faisceau de rayons X dans \textbf{Chapitre~\ref{sec:effect_optical_imperfections}}. Ce chapitre commence par énumérer les étapes importantes des techniques de fabrication additive et soustractive extrêmement précises qui ont permis de produire des optiques à forme libre très précises pour la correction des aberrations optiques. Un examen des stratégies de correction des imperfections optiques dans l'optique des rayons X est donné et une méthode méthodique de calcul de la plaque de phase de correction réfractive est présentée. En utilisant les outils de simulation développés pour cette thèse, la performance attendue de la plaque de correction peut être calculée et des tolérances d'alignement peuvent être dessinées. Un prototype en diamant est présenté et un test à l'oreille sur un faisceau de rayons X est montré pour démontrer une amélioration qualitative du profil du faisceau. Une longue discussion sur la conception et les performances attendues par rapport aux premiers essais de plaque sur un faisceau de rayons X est présentée à la fin de ce chapitre.
\\
\\
\textbf{Chapitre \ref{sec:conclusion_en}~-\nameref{sec:conclusion_en}} résumant la signification, les implications, les contributions et les limites de cette thèse et exposant les orientations futures.
\\
\\
\textbf{Annexe \ref{sec:publications}~-~\nameref{sec:publications}} des travaux produits par le candidat au doctorat au cours de ce projet.