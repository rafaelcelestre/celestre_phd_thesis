% 13/11/2020 - First round of corrections: added Thomas & Ray corrections
\pdfbookmark[0]{Abstract}{Abstract}
\chapter*{Abstract}
\label{sec:abstractEN}
\vspace*{-10mm}

The advent of the 4$^{\text{th}}$ generation high-energy synchrotron facilities (ESRF-EBS and the planned APS-U, HEPS, PETRA-IV and SPring-8 II) and free-electron lasers (Eu-XFEL, SLAC) allied with the recent demonstration of high-quality free-form refractive optics for beam shaping and optical correction has reinforced interest in compound refractive lenses (CRLs) as optics for beam transport, probe formation in X-ray micro- and nano-analysis as well as for imaging applications. Within this context, in 2016, the ESRF resumed the fabrication and tests of 2D focusing bi-concave aluminium X-ray lenses. Current refractive optics, commercial or otherwise, have non-negligible aberrations which deteriorate their final performance and accurate modelling with input from metrology is necessary to evaluate the wavefront degradation and in order to propose mitigation strategies.

In physical optics, weakly focusing elements are usually simulated as a single thin element in the projection approximation. While a single X-ray lens at typical operation conditions can often be represented in this way, simulating a full CRL with a similar approach leads to an idealised model that lacks versatility. This work proposes decomposing a CRL into its lenslets separated by a free-space propagation, which resembles the multi-slicing techniques (MS) already used for optical simulations. Attention is given to modelling the single lens element by adding additional degrees of freedom allowing the modelling of typical misalignments and fabrication errors. Orthonormal polynomials for optical aberrations as well as at-wavelength metrology data are also used to obtain realistic simulation results, which are presented in several coherent- and partially-coherent simulations throughout this work. They compare qualitatively well with the experimental data and are used to evaluate the effect of optical imperfections on partially coherent X-ray beam and the suitability of common figures of merit. Unlike other works, the modelling presented here can be used transparently with one of the most popular codes for X-ray beamline design, "Synchrotron Radiation Workshop" (SRW), and is publicly available on GitLab.

Optical imperfections measured with high spatial resolution can be added to the MS representation of a CRL to accurately represent real X-ray lenses. X-ray speckle vector tracking (XSVT) is a versatile technique that conveniently obtains the figure errors of X-ray lenses in the projection approximation with high spatial resolution and is used in this work for characterising lenses to be used in the modelling presented here. This thesis presents a review of most relevant concepts of X-ray speckle based imaging applied to lens metrology.

Implementing the MS model of a CRL using metrology data allows extraction of the accumulated figure errors of stacked lenses and enables the calculation of phase corrections. Finally, this thesis presents a methodology for calculating the profile of refractive correctors, which is applied to produce phase plates ablated from diamond. Early experimental results show an improvement on the beam profile, but the transverse alignment of the phase-plate is a limiting factor. Further improvements to the metrology of lenses and correction plates and alignment protocols are necessary to optimise the performance of these optical correctors.