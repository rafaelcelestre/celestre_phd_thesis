% \appendix

\chapter*{Foreword}\addcontentsline{toc}{chapter}{Foreword}

Improvements in the quality of modern X-ray sources place increasingly stringent demands upon the quality of X-ray optics. X-ray optics used to focus X-ray beams should not degrade the X-ray source brilliance. To achieve this the perturbation of the X-ray wavefront, adverse effects on the focal spots, and intensity losses all need to be minimized. For CRL this requires lens shape fidelity, smooth surfaces and a homogeneous pure internal structure. This PhD project aims at measuring the properties of  X-ray optics and compare this to computer simulations that take into consideration the source size and degree of coherence expected for the upgraded, EBS storage ring. It has therefore both an experimental and a simulation part. It will allow measurement of the effect on the X-ray beam of each individual lens via near-field speckle based imaging or comparable techniques, prediction of the effect of a CRL stack on the beam, and verification of the focusing properties of the stack with the X-ray beam. Initially, this will be possible only with the less coherent current ESRF beam, but tests in the more coherent beam of the EBS should be possible in the third year. This project addresses important aspects of the EBS X-ray optics R$\&$D programme as laid out in the Orange book.

\section*{Motivation}\addcontentsline{toc}{section}{Motivation}

This PhD project aims at determining the effect of real, imperfect X-ray optics on a partially coherent X-ray beam, characteristic of the ESRF source after the EBS upgrade. In particular, the effect of refractive lens shape errors, surface roughness, and impurities on a partially coherent X-ray beam shall be simulated. In order to obtain realistic results, experimental speckle-tracking wavefront sensing techniques shall be used to characterise X-ray lenses and other optics. Novel lenses currently under development in the X-ray optics group are 2D focusing kinoform lenses in SU-8 polymers, 1 and 2D diamond as well as in-house fabricated Al-lenses. 3D printing and other fabrication techniques will be investigated to determine their suitability for use in the manufacture of corrective optics for imperfect lens or mirror systems.

\section*{Objectives}\addcontentsline{toc}{section}{Objectives}

\begin{itemize}
\item Wave-optics simulations: simulations using available tools (SRW, COMSYL, Shadow, ...) on partially coherent beams passing through 2D focusing X-ray lenses featuring realistic levels of imperfections: a) shape errors, b) surface roughness, c) inclusions and voids. The influence of optical imperfections shall be investigated for a partially coherent beam and the possibility to use corrective optics components to mitigate some of the wavefront degradation will be explored;

\item Study the limitations of the thin object approximation used in most simulation codes for dealing with the optical element imperfections. Evaluate its impact on the simulations for the imperfection in CRL. If necessary, develop methods to go beyond this approximation, for example by solving the Maxwell equations for a surface with imperfections illuminated by an X-ray wavefront via Finite Element Methods;
	
\item Speckle based imaging for wavefront sensing measurements on real lenses and arrays with the aim to correct shape errors with corrective optics produced by various fabrication methods (e.g. 3D printing, FIB machining, etc).
	
\end{itemize}

\section*{Outline}\addcontentsline{toc}{section}{Outline}

This work is divided into six chapters, discussion and a conclusion. The relevant publications during this PhD project are summarised in an appendix at the end of this work. This PhD thesis is organised as follows:
\\
\\
\textbf{Chapter \ref{sec:x-rays}~-~\nameref{sec:x-rays}}, a nod to the publication of the discovery of X-rays, is the first of two theoretical chapters, presents ...
\\
\\
\textbf{Chapter \ref{sec:x-ray_optics}~-~\nameref{sec:x-ray_optics}}, a reference to A. Compton's Nobel lecture, introduces ... this chapter concludes the presentation of the theoretical aspects necessary for this dissertation.
\\
\\
\textbf{Chapter \ref{sec:modelling}~-~\nameref{sec:modelling}} introduces ...
\\
\\
\textbf{Chapter \ref{sec:measuring}~-~\nameref{sec:measuring}} introduces ...
\\
\\
\textbf{Chapter \ref{sec:effect_optical_imperfections}~-~\nameref{sec:effect_optical_imperfections}} introduces ...
\\
\\
\textbf{Chapter \ref{sec:corrections}~-~\nameref{sec:corrections}} introduces ...
\\
\\
\textbf{Chapter \ref{sec:conclusion_en}~-~\nameref{sec:conclusion_en}}  ...
\\
\\
\textbf{Appendix \ref{sec:publications}~-~\nameref{sec:publications}}  ...

$\blacksquare$