
\pdfbookmark[0]{Acknowledgements}{Acknowledgements}
\chapter*{Acknowledgements}
\label{sec:acknowledgement}
\vspace*{-10mm}

Working on my PhD project and bringing it to a conclusion would not have been possible without the help of several people. I would like to take a moment to thank them while having in mind that when naming so many people, there is always the risk of missing a name or two. To that end, if your name is missing, I apologise.

I start this long list by thanking my PhD supervisors Manuel Sanchez del Rio - whom I met way back in 2015: long before my PhD project started - and Thomas Roth. While \textit{Manolo} helped me getting started with my simulations and introduced me to prominent researchers in my field, Thomas was a great company during several beamtimes - more often than not, until very early in the morning - in Grenoble and Chicago. Both Manolo and Thomas were always open for discussions at every stage of the research project while allowing this PhD thesis to be my work. Similarly, I would like to extend my sincere thanks to Ray Barrett, who helped to steer this work in the right direction whenever he thought I needed it.

I also want to express my gratitude to the members of my jury: Lucia Alianelli, Vincent Favre-Nicolin, Chris Jacobsen, Jose Emilio Lorenzo Dias, David Paganin and Christian Schroer for accepting reviewing and reporting on my work. Our exchanges were very rich and it was a pleasure to expose my work to them.

This work made extensive use of experimental data and I must thank the staff of the BM05 and ID06 beamlines at the ESRF. Special thanks to Sebastien Berujon (BM05) and Carsten Detlefs (ID06). Always very attentive, \textit{Seb} introduced me to wavefront sensing with speckles and helped me gain proficiency at the beamline. Carsten is an early enthusiast of the project - always keen on providing insightful feedback and accommodating Thomas and me on the tight beamline schedule. I would be remiss if I didn’t thank our colleagues from the 1-BM/X-ray Optics Group from the Advanced Photon Source - Argonne National Laboratory (APS/ANL), who hosted our experiments during the long ESRF shutdown for the ESRF-EBS upgrade: Lahsen Assoufid, Xianbo Shi, Zhi Qiao, Michal Wojcik and their great technical staff. Also, thanks to Carsten Detlefs; Terence Manning; Sergey Antipov; Arndt Last \& Elisa Konnermann; Christian David \& Frieder Koch;  and Paw Kristiansen for providing samples for the beamtimes.

I would like to acknowledge Oleg Chubar, whom I met while still working on my master's thesis in 2017. Oleg has often made himself available for our numerous discussions - always very patient when explaining the ins and outs of simulations with SRW, physical optics or accelerator physics. This good synergy can be exemplified by the two visits to the National Synchrotron Light Source-II/Brookhaven National Laboratory, where Oleg and I could work closely together. We acknowledge the ”DOE BES Field Work Proposal PS-017 funding” for partially financing these two stays. I should also thank Luca Rebuffi for, among other things, showing interest in my work and for creating a \textit{pip-installable} version of my python libraries to be distributed with OASYS.

A general shout out to all the people I have had fruitful discussions with, E-mails exchanges or (watered down) coffees during conference breaks: Sajid Ali, Ruxandra Cojocaru, Vincent Favre-Nicolin, Wallan Grizolli, Herbert Gross, Chris Jacobsen, Michael Krisch, Arndt Last, S\'ergio Lordano, Virendra Mahajan, Bernd Meyer, Peter Munro, Boaz Nash, David Paganin, Maksim Rakitin, Claudio Romero, Andreas Schropp, Frank Seiboth, Irina Snigireva, Pedro Tavares (…)

My research was conducted within the X-ray Optics Group (XOG) from the Instrumentation Services \& Development Division (ISDD) at the European Synchrotron (ESRF) and I would like to thank my colleagues, especially the XOG, for providing a very nice work environment. Amparo Vivo merits a special thanks for proof-reading the French parts of my thesis.

I do not know when I realised I wanted to have a career in science, but for as long as I can remember, \textit{I wanted to be a lumberjack}, I mean, a scientist. Following that dream would not have been possible without the support and incentive of my parents, grandparents \& my family. A special mention to my friends from Rep\'ublica Beijos me Liga and Coloc la Flemme - too many to mention by name; to Baptiste Belescot, Gaetan \textit{PC} Girard, Edoardo Zatterin, Luca Capasso, Rafael Vescovi, Jan Sandner, Maik Rosenberger and the Konnermanns. These are the people who have been there for me since I left Brazil and I am thankful.

Finally, there is a quotation by Joe Walsh that sums up very well how writing this PhD thesis was like to me: "\textit{I can't complain, but sometimes I still do. Life's been good to me so far}".

\vfill
\begin{flushright}
	\begin{minipage}{5cm}
	    \centering Muito obrigado,\\
		\centering\thesisName
	\end{minipage}
\end{flushright}
\vfill