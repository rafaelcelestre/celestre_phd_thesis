\chapter*{Préface}\addcontentsline{toc}{chapter}{Préface}

L'amélioration de la qualité des sources de rayons X modernes impose des optiques pour rayons X de qualité considérablement accrue, permettant de conserver la brillance de la source même focalisée.  Pour cela, il faut réduire au minimum la perturbation du front d'onde des rayons X, les effets néfastes sur les points focaux et les pertes d'intensité. Pour y parvenir avec des lentilles rayons X, elles doivent présenter une fidélité de forme, des surfaces lisses et une structure interne homogène et pure. 

Ce projet de doctorat visait à déterminer l'effet des erreurs de forme des lentilles réfractives, de la rugosité de surface et des impuretés sur un faisceau de rayons X partiellement cohérent ayant des caractéristiques similaires à celles d’une ligne de lumière avec onduleur après la mise à niveau de l'ESRF-EBS. Sur la base de développements récents, l'atténuation des erreurs de forme des lentilles à l'aide d'optiques correctives a également été étudiée. Pour atteindre les objectifs proposés, ce projet reposait sur deux piliers : théorique et expérimental, avec des aspects techniques liés aux deux. Ce projet a abordé des aspects importants du programme de R\&D en optique des rayons X de l'ESRF-EBS, tel qu'il est défini dans le plan stratégique de mise à niveau (\textit{Orange book}).

Le volet théorique du travail a consisté à étudier les limites de la modélisation actuelle et les approximations utilisées dans la plupart des codes de simulation pour traiter les éléments optiques. Après avoir évalué la validité des outils existants, des propositions d'extension et de nouveaux développements ont été proposés. Les objectifs étaient également, d’une part, d'ajouter aux simulations la capacité de traiter les données de métrologie et de développer un cadre pour la conception d'optiques réfractives correctives, et d’autre part, d’intégrer les modèles de simulation à des simulations cohérentes et partiellement cohérentes pour obtenir de manière réaliste l'effet des imperfections optiques sur un faisceau de rayons X et de comparer les résultats avec la littérature et les données expérimentales. Les objectifs techniques liés à cette partie théorique comprenaient le développement de bibliothèques Python permettant d'utiliser facilement la modélisation nouvellement développée avec le code "Synchrotron Radiation Workshop" (SRW) pour la conception des lignes de faisceaux. Une partie de ce développement a été mené en collaboration avec O. Chubar (auteur du SRW) au cours de deux visites scientifiques au Brookhaven National Lab. aux États-Unis. Cette nouvelle bibliothèque Python est disponible pour une intégration dans des interfaces graphiques utilisateur telles que OrAnge SYnchrotron Suit (OASYS). 

Afin d'obtenir des résultats de simulation réalistes, des techniques de détection de front d'onde en champ proche basées sur le speckle des rayons X ont été utilisées pour caractériser les lentilles produites en interne, les lentilles et optiques de type « free-form » dans le cadre de collaborations scientifiques, et les lentilles commerciales récemment acquises. La métrologie dite « en longueur d'onde » a été effectuée sur la ligne de lumière BM05 de l'ESRF jusqu’à son arrêt début décembre 2018, puis sur la ligne 1-BM de l'APS à Chicago en 2019 et enfin sur la ligne ID06 pendant la période de mise en service de l'ESRF-EBS. Outre la mesure de composants optiques pour rayons X et la création d'une base de données de métrologie pour les lentilles rayons X, les objectifs techniques de cette partie expérimentale étaient de former le doctorant pour le rendre autonome et compétant dans la mise en œuvre du dispositif expérimental sur une ligne de lumière, mais aussi dans l’acquisition et le traitement des données. Le développement de protocoles d’alignement, de standardisation des mesures et d’analyse des données étaient également attendus. 

Par la suite, l’étude de récents développements concernant les techniques de fabrication additive et soustractive pour la réalisation d’éléments de correction optique s’est avérée nécessaire. C’est donc naturellement que cette formation doctorale s’est achevée par la conception d’un premier correcteur de phase suivi d’expériences pour évaluer ses performances sur le faisceau de rayons X à l'ESRF.


\section*{Aperçu}\addcontentsline{toc}{section}{Aperçu}

Ce travail est divisé en six chapitres, une conclusion et une annexe résumant les publications pertinentes de l'auteur au cours de ce projet de doctorat :
\\
\\
\textbf{Chapitre \ref{sec:x-rays}~-~\nameref{sec:x-rays}} est le premier de deux chapitres essentiellement théoriques. Le titre est un clin d’œil au titre de la publication qui relate la découverte des rayons X en 1895. Il commence par introduire les concepts de brillance et les relie aux sources de rayons X basées sur les accélérateurs. Les sources de rayons X à haute brillance sont ensuite présentées et le concept de rayons X latéralement cohérents est illustré. L’optique physique est ensuite présentée comme la description la plus appropriée des champs (partiellement) cohérents. La propagation en espace libre et l'approximation paraxiale sont expliquées et la propagation des rayons X à travers la matière est modélisée par l'introduction de l'élément de transmission. Une brève discussion sur la cohérence optique et la présentation de quelques concepts de base sont faites à la fin de cette section. Ce chapitre se termine par une discussion sur les simulations, méthodes et approches pour optiques des rayons X.
\\
\\
\textbf{Chapitre \ref{sec:x-ray_optics}~-~\nameref{sec:x-ray_optics}}, en référence au discours de A. Compton lauréat du prix Nobel. Ce chapitre s'ouvre sur un récit historique des débuts de la science des rayons X montrant les étapes qui ont conduit à la compréhension des rayons X comme branche de l'optique. Un bref examen des développements de l'optique de focalisation des rayons X en fonction des phénomènes optiques est donné pour contextualiser l'évolution récente de l'optique réfractive. La modélisation d’une lentille rayons X idéale et celle d’un empilement idéal, basé sur des techniques de type multi-coupes, sont présentés. Avec peu de modifications, ce modèle peut accepter des cartographies d'erreurs arbitraires pour tenir compte des imperfections optiques. Des mesures déterminantes pour l'évaluation des performances de CRL sont introduites et les conditions de tolérance des aberrations sont présentées. Ce chapitre conclut la présentation des aspects théoriques nécessaires à cette thèse.
\\
\\
\textbf{Chapitre \ref{sec:modelling}~-~\nameref{sec:modelling}} se concentre sur la modélisation d’une lentille rayons X. En ajoutant des degrés de liberté latéraux et angulaires aux faces avant et arrière des éléments focalisants, il est possible d’imiter les désalignements et les erreurs de fabrication typiques rencontrés dans les lentilles réelles. Dans les cas où ces degrés de liberté paramétrés ne suffisent pas, une modélisation d’erreurs de forme plus complexe est possible en utilisant les polynômes orthonormaux Zernike ou 2D Legendre, ou encore des données de métrologie. Ce chapitre se termine par le détail des bibliothèques Python implémentées pour la modélisation des imperfections de phase pour les lentilles rayons X.
\\
\\
\textbf{Chapitre \ref{sec:measuring}~-~\nameref{sec:measuring}} présente une description complète de la technique de suivi des vecteurs de speckle en champ proche par rayons X (XSVT) employée pour inspecter les lentilles utilisées dans cette thèse. Il commence par décrire la diversité des techniques de métrologie en longueur d’onde et explique pourquoi la technique XSVT est la plus appropriée pour ce travail. Un examen des principaux aspects du dispositif expérimental, de l’acquisition, du traitement et de l’analyse des données est présenté et une discussion sur la métrologie des lentilles rayons X par rapport aux empilements de lentilles clôt ce chapitre.
\\
\\
\textbf{Chapitre \ref{sec:effect_optical_imperfections}~-~\nameref{sec:effect_optical_imperfections}} traite les effets induits par les imperfections optiques sur la dégradation du faisceau de rayons X en présentant de nombreuses simulations des caustiques du faisceau, de la fonction d’étalement du point et du profil du faisceau à des positions déterminées le long de l’axe optique pour différentes configurations optiques. Une étude comparative des simulations de la métrologie des lentilles individuelles par rapport aux lentilles empilées est présentée, suivie d’une discussion sur l’effet des imperfections optiques et la pertinence du rapport de Strehl pour les lentilles rayons X. Quelques commentaires sur les temps de simulation concluent le chapitre.
\\
\\
\textbf{Chapitre \ref{sec:corrections}~-~\nameref{sec:corrections}} est le dernier chapitre et conclut ce voyage qui a commencé dans le \textbf{Chapitre~\ref{sec:x-ray_optics}} par la modélisation des lentilles idéales, en passant par la modélisation des lentilles aberrantes dans le \textbf{Chapitre~\ref{sec:modelling}}, en mesurant leurs imperfections dans le \textbf{Chapitre~\ref{sec:measuring}} et en comprenant leurs effets sur un faisceau de rayons X dans le \textbf{Chapitre~\ref{sec:effect_optical_imperfections}}. Ce chapitre commence par énumérer les étapes importantes des techniques extrêmement précises de fabrication additive et soustractive qui ont permis de produire des optiques « free-form » très précises pour la correction des aberrations optiques. Un récapitulatif des stratégies de correction des imperfections optiques pour les rayons X est donné et une approche méthodique du calcul de la plaque réfractive de correction de phase est présentée. Les outils de simulation développés pour cette thèse ont permis d’évaluer la performance attendue de la plaque corrective ainsi que les tolérances d’alignement.  Un prototype en diamant est présenté et une première expérience sur une ligne de lumière démontre une amélioration qualitative du profil du faisceau de rayons X. Une longue discussion sur la conception et les performances attendues confrontées aux résultats expérimentaux obtenus avec la plaque corrective sur le faisceau de rayons X est présentée à la fin de ce chapitre.
\\
\\
\textbf{Chapitre \ref{sec:conclusion_fr}~-\nameref{sec:conclusion_fr}} résumant la signification, les implications, les contributions et les limites de cette thèse et exposant les orientations futures.
\\
\\
\textbf{Annexe \ref{sec:publications}~-~\nameref{sec:publications}} liste les publications réalisées par le doctorant au cours de ce projet.