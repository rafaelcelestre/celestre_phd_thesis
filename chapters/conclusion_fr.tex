\begin{refsection}\renewcommand{\thechapter}{7.fr}
\chapter{~~Conclusions}\label{sec:conclusion_fr}

La quatrième génération d'anneaux de stockage, récemment lancée, impose des exigences strictes quant à la qualité des éléments optiques afin de minimiser la dégradation du faisceau de rayons X [\cite{Schroer2014,Yabashi2014}]. L'objectif de cette thèse, qui consiste à étudier et à modéliser l'effet des imperfections optiques des lentilles réfractives composées sur un faisceau de rayons X partiellement cohérent, aborde des aspects importants du programme de R\&D en optique des rayons X de l'ESRF-EBS, tel qu'il est exposé dans le livre orange [\cite{orangebook}]. 

Sur la base des concepts d'optique physique présentés au chapitre~\ref{sec:x-rays} et des éléments optiques déjà mis en œuvre dans le SRW [\cite{Baltser2011}], un modèle élargi de lentille réfractive composée (LRC) idéale épaisse a été présenté. Ce modèle est similaire à la technique de découpage en plusieurs tranches déjà utilisée pour la modélisation optique [\cite{Li2017,Ali2020}], où chaque lentille est considérée comme une tranche du LRC. L'élément de transmission complexe représentant un LRC épais est donné par Eq.~\ref{eq:TE_CRL_MS} et illustré à la Fig.~\ref{fig:models}(b). Ce modèle idéal d'un LRC épais peut être modifié pour tenir compte des imperfections optiques lorsqu'une carte en 2D des erreurs de figure de chaque lentille est disponible. Ce modèle est illustré à la Fig.~\ref{fig:models}(c) et décrit par Eq.~\ref{eq:TE_CRL_MS_ERR}. Cette modélisation du LRC en tant qu'élément optique épais devient plus pertinente lorsque la pile de lentilles est composée d'un grand nombre de petites lentilles et qu'il y a une focalisation importante à l'intérieur du LRC [\cite{Schroer2005}]. La modélisation étendue des LRC a été publiée dans [\cite{Celestre2020}].

Les erreurs de fabrication couramment présentes dans les lentilles à rayons X en relief (Fig.~\ref{fig:lens_cuts}) ont été paramétrées en accordant plus de degrés de liberté au modèle idéal pour une seule lentille décrit dans [\cite{Baltser2011}]. Cette paramétrisation est suffisamment générale pour s'appliquer aux lentilles produites par d'autres techniques, mais cette modélisation pourrait bénéficier de l'ajout du choix de la section conique à utiliser, actuellement limitée au cas parabolique, comme cela a déjà été mis en œuvre par [\cite{SanchezdelRio2012,Andrejczuk2010}] pour le ray-tracing. Certains groupes ont déjà expérimenté des formes non paraboliques pour la focalisation des rayons X [\cite{Alianelli2007,Evans-Lutterodt2003, Alianelli2015, Sutter2017}] et la disponibilité d'outils de simulation pourrait raviver l'intérêt pour les conceptions non orthodoxes. Il serait malvenu de ne pas mentionner les similitudes entre la modélisation présentée ici et les travaux présentés par [\cite{Andrejczuk2010}], où la modélisation du rôle des erreurs d'un seul élément dans les lentilles à rayons X est mise en œuvre pour le traçage des rayons, mais l'analyse est limitée à la largeur et à l'intensité du faisceau dans le plan image. La modélisation d'erreurs de forme plus complexes a été rendue possible par l'utilisation des polynômes orthonormaux de Zernike ou de Legendre 2D. Cela a été fait en unifiant certaines parties déjà existantes des bibliothèques Python, en ajoutant de nouveaux ensembles de polynômes et en les interfaçant pour qu'ils soient compatibles avec le cadre déjà décrit. Cette bibliothèque Python unifiée et étendue a été utilisée tout au long de la thèse pour ajuster les erreurs de figures présentées tout au long des chapitres~\ref{sec:modelling}-\ref{sec:corrections}. La modélisation des erreurs de forme, soit en accordant plus de degrés de liberté au modèle de lentille idéal, soit en générant directement les erreurs de forme par des polynômes orthonormaux, est un outil utile pour délimiter les tolérances sur la fabrication des lentilles radiographiques, ce qui est particulièrement pertinent pour le projet de fabrication en interne de lentilles en aluminium. Toute carte 2D arbitraire des imperfections de surface (par exemple, des données de métrologie ou des surfaces de forme libre pour la mise en forme du faisceau) peut également être intégrée dans les simulations à l'aide de ce cadre. Plusieurs auteurs ont développé de nombreux outils avec différents degrés de complexité pour simuler les lentilles à rayons X et leurs aberrations. Toutefois, non seulement ces outils ne sont pas accessibles au public, mais la modélisation n'est souvent pas compatible avec les outils de simulation déjà largement répandus pour la conception optique des rayons X, tels que SHADOW (traçage des rayons) et SRW (propagation du front d'onde). La modélisation présentée ici est open source, adaptée pour être utilisée de manière transparente avec SRW et est actuellement disponible dans un dépôt public au GitLab jusqu'à la fusion éventuelle avec la distribution officielle de SRW. Parallèlement, le code est incorporé dans la boîte à outils OASYS [\cite{Rebuffi2017}] qui sera disponible pour les distributions de SHADOW, du ray-tracing hybride et de SRW. Ces développements récents sur la modélisation des lentilles à rayons X ont été présentés dans [\cite{Celestre2020b}] et utilisés dans certaines des simulations de [\cite{Chubar2020}].

Finalement, pour les simulations avec des niveaux d'imperfections réalistes, la modélisation des erreurs de chiffres sur les lentilles radiographiques individuelles a été faite avec des données de métrologie sous forme de cartes 2D des déviations locales du profil parabolique. La technique de métrologie utilisée pour cette thèse était le XSVT (X-ray vectorial speckle tracking). En plus d'être relativement simple à mettre en œuvre, cette technique possède une résolution latérale élevée et une bonne sensibilité. De plus, elle fournit une carte 2D des erreurs de figures dans l'approximation de la projection, qui peut être facilement utilisée avec les simulations optiques d'ondes. 
Plusieurs campagnes de mesures ont été effectuées à l'ESRF, à l'APS et à l'ESRF-EBS pour évaluer la qualité des lentilles produites en interne, des lentilles et des optiques à forme libre dans le cadre de collaborations scientifiques et des lentilles commerciales nouvellement acquises. En conséquence, une base de données importante et diversifiée de fichiers de métrologie a été constituée. Après un certain temps de maturation, une expansion possible de la DABAM (base de données pour la métrologie des miroirs à rayons X) [\cite{SanchezDelRio2016}] est envisagée. La mise à disposition des données de métrologie des miroirs à rayons X par le biais d'une base de données open-source aiderait à consolider le cadre développé ici puisque la DABAM est également distribuée et interfacée par OASYS. Avant cela, cependant, il faut s'efforcer de faire converger la métrologie d'une lentille unique et le jalonnement ultérieur par logiciel avec la métrologie d'un empilement de lentilles composé de ces mêmes lentilles. L'état actuel montre un certain accord qualitatif (Fig.~\ref{fig:accumulated_profile_1}-\ref{fig:CDo}), qui est transféré aux simulations présentées dans les Figs.~\ref{fig:CDn_vs_CDnStack}-\ref{fig:CDo_vs_CDoStack} et Fig.~\ref{fig:CD_Strehl}. Cela est suffisant pour les études préliminaires mais laisse une marge de manœuvre pour améliorer et délimiter les protocoles de métrologie, tant en termes d'alignement des échantillons que de traitement des données - ce qui devrait être l'objectif des travaux futurs. L'importance majeure de pouvoir empiler de manière fiable des lentilles mesurées individuellement est la possibilité d'évaluer les performances d'un LCR composé de n'importe quelle sélection arbitraire de lentilles hors ligne, avec une application immédiate à la conception de correcteurs optiques. L'expérience acquise lors des campagnes de métrologie et les progrès réalisés dans le post-traitement des données acquises ont permis d'apporter de modestes contributions à [\cite{Berujon2020a,Berujon2020,Qiao2020b}].

Les simulations, présentées dans la première partie du chapitre ~\ref{sec:effect_optical_imperfections} et résumées par la figure ~\ref{fig:CDnS}, montrent un faisceau caustique qui a été signalé plusieurs fois dans la littérature pour le même type de lentilles à rayons X [\cite{Schropp2013,Seiboth2017,Seiboth2020}]. La mesure expérimentale des caustiques de faisceau pour cet empilement de lentilles montre également une similarité avec ce qui a été prédit par les simulations - cf. Fig.~\ref{fig:experimental_CDn_pp}. La similitude entre les simulations et les données expérimentales aide à valider le cadre présenté dans cette thèse et on pense qu'elles permettent d'évaluer qualitativement les effets des imperfections optiques sur la dégradation du profil du faisceau de rayons X. Les simulations permettent également d'étudier l'adéquation du rapport de Strehl comme chiffre de mérite pour la conception optique de systèmes fonctionnant loin du critère de Maréchal pour la qualité optique. En général, les équations utilisées pour estimer le rapport de Strehl tendent à sous-estimer la performance du système dans cette région (Fig.~\ref{fig:Strehl}). L'échec de la tentative d'obtenir des équations de remplacement simples (cf. Eq.~\ref{eq:Strhel_Exp}) indique que la manière la plus simple d'évaluer le rapport de Strehl est la simulation optique du système en utilisant des erreurs de chiffres réalistes. Il est également pertinent de mentionner que bien que les simulations entièrement et partiellement cohérentes ne diffèrent pas de manière significative, il est prématuré de généraliser ce comportement car les simulations partiellement cohérentes ont été réalisées pour une ligne de faisceaux ondulatoire hypothétique fonctionnant avec le réseau magnétique ESRF amélioré. Un autre résultat des simulations est la prédiction que les systèmes avec des erreurs de figures à haute fréquence spatiale ne présentent pas un élargissement de la taille du faisceau au plan focal, mais que l'intensité est plutôt réduite par la diffusion des photons autour du lobe principal, ce qui augmente le niveau de fond - Fig.~\ref{fig:hf_strehl_scan}. Les effets des erreurs de figures à haute fréquence spatiale sont importants, car des systèmes optiques bien corrigés en feraient une source prédominante d'erreurs de forme. Cette analyse a été publiée dans [\cite{Celestre2020}].


La modélisation multi-coupes d'un LCR incluant les données métrologiques des lentilles mesurées individuellement (Eq.~\ref{eq:TE_CRL_MS_ERR}) permet l'extraction des erreurs de chiffres cumulées d'un empilement de lentilles. Ceci a été utilisé pour modéliser une plaque de phase en diamant afin de corriger ces erreurs à la pupille de sortie du système. En raison des limites de l'alignement des plaques de phase dans une ligne de faisceau, la conception symétrique est préférée au détriment de la performance de correction - les résultats rapportés montrent une excellente performance même avec ce compromis [\cite{Seiboth2020,Dhamgaye2020}]. Le profil du faisceau illustré à la figure ~\ref{fig:CDn_corrected} est plus homogène en amont et en aval du plan image et ressemble aux simulations pour un système à prédominance de hautes fréquences spatiales à la figure ~\ref{fig:CDnHF}. Cependant, lors de l'analyse du rapport de Strehl, la performance de la plaque de phase modélisée est limitée et dépasse ce que d'autres groupes ont rapporté. Les principales raisons de ces divergences sont le fait que le profil à corriger (Fig.~\ref{fig:accumulated_profile_1}) comporte des composantes non symétriques importantes (aberration en trèfle) en plus de la prédominance classique des aberrations sphériques. L'absence d'asymétrie constatée par d'autres groupes est probablement liée au fait qu'un plus grand nombre de lentilles sont utilisées dans leurs piles et l'ajout d'erreurs corrélées de lentilles à rotation aléatoire aurait pour effet de donner une forme plus symétrique aux erreurs accumulées en faisant la moyenne des composantes non symétriques. La rotation aléatoire des lentilles dans les simulations pourrait contribuer à améliorer le rapport de Strehl attendu et une procédure similaire pourrait être appliquée à la pile de lentilles dans le cadre d'une expérience. Les plaques de phase conçues ont été commandées à un partenaire commercial et elles ont été enlevées du diamant par un laser femtoseconde. Les premiers essais sur un faisceau de rayons X ont montré que les plaques de phase n'étaient pas centrées dans le cadre dans les tolérances calculées, ce qui est crucial pour leur performance. La plaque de phase devait être alignée par rapport à la pile de lentilles, ce qui a posé un problème initial, car à l'époque, aucun protocole d'alignement n'avait été défini. Cela a également mis en évidence la nécessité de réimplémenter des techniques de détection de front d'onde plus rapides, qui ont été abandonnées au profit d'une résolution spatiale plus élevée. D'autres problèmes tels que la surexposition du détecteur même après une forte atténuation du faisceau ont limité les tests qui pouvaient être effectués avec la plaque de correction, de sorte qu'aucune valeur quantitative n'a été générée pour évaluer la performance de la correction, cependant, une caustique du faisceau a pu être enregistrée pour le système aberré et corrigé (Fig.~\ref{fig:experimental_CDn_pp}) montrant qualitativement un faisceau plus homogène surtout au voisinage du plan focal, avec une meilleure performance en aval. Les premiers résultats sont prometteurs mais montrent la nécessité de replanifier les temps de faisceau suivants et de chercher d'autres moyens d'extraire rapidement le front d'onde résiduel et d'aligner la plaque de correction, de mesurer la cause du faisceau et d'évaluer quantitativement les performances de correction. L'exploration de la fabrication additive pour la conception d'optiques de correction dans d'autres matériaux devrait également être étudiée comme alternative au diamant et pour réduire les coûts de prototypage. $\blacksquare$

\addcontentsline{toc}{section}{References}
\printbibliography[heading=subbibliography]
\end{refsection}