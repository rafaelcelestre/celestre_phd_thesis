\begin{refsection}\renewcommand{\thechapter}{7.fr}
\chapter{~~Conclusion}\label{sec:conclusion_fr}

La quatrième génération d'anneaux de stockage, récemment lancée, impose des exigences strictes quant à la qualité des éléments optiques afin de minimiser la dégradation du faisceau de rayons X  [\cite{Schroer2014,Yabashi2014}]. Cette thèse, dont l'objectif consiste à étudier et à modéliser l'effet des imperfections optiques des lentilles réfractives composées sur un faisceau de rayons X partiellement cohérent, aborde des aspects importants du programme de R\&D en optique des rayons X de l'ESRF-EBS, tel qu'il est exposé dans le \textit{Orange book} [\cite{orangebook}]. 

Basé sur les concepts d'optique physique présentés au chapitre~\ref{sec:x-rays} et les éléments optiques disponibles dans SRW [\cite{Baltser2011}] un modèle étendu à une lentille réfractive composée (CRL) idéale et épaisse a été présenté. Ce modèle s’apparente à la technique de découpage multi-coupes déjà utilisée pour la modélisation optique [\cite{Li2017,Ali2020}], où chaque lentille est considérée comme une tranche de la CRL. L'élément de transmission complexe représentant une CRL épaisse est donné par Eq.~\ref{eq:TE_CRL_MS} et illustré par la Fig.~\ref{fig:models}(b). Ce modèle idéal d'une CRL épaisse peut être modifié pour prendre en compte des imperfections optiques lorsqu'une cartographie 2D des erreurs de forme de chaque lentille est disponible. Il est illustré par la Fig.~\ref{fig:models}(c) et décrit par Eq.~\ref{eq:TE_CRL_MS_ERR}. Cette modélisation de la CRL en tant qu'élément optique épais devient plus pertinente lorsque l’empilement est composé d'un grand nombre de lentilles et que la focalisation au sein du CRL est importante [\cite{Schroer2005}]. La modélisation étendue des CRL a été publiée dans [\cite{Celestre2020}].

Les erreurs de fabrication couramment présentes dans les lentilles rayons X en relief (Fig.~\ref{fig:lens_cuts}) ont été paramétrées en offrant plus de degrés de liberté au modèle idéal pour une seule lentille comme décrit dans [\cite{Baltser2011}]. Ce paramétrage est assez général pour s'appliquer aux lentilles produites par d'autres techniques, mais cette modélisation pourrait être améliorée en offrant le choix de la section conique à utiliser, actuellement limitée au cas parabolique, comme cela a déjà été implémenté pour le ray-tracing par [\cite{SanchezdelRio2012,Andrejczuk2010}]. Certains groupes ont déjà expérimenté des formes non paraboliques pour la focalisation des rayons X [\cite{Alianelli2007,Evans-Lutterodt2003, Alianelli2015, Sutter2017}] et la disponibilité d'outils de simulation pourrait raviver l'intérêt pour les conceptions non orthodoxes. Il serait inconvenant de ne pas mentionner les similitudes entre la modélisation présentée ici et les travaux présentés par [\cite{Andrejczuk2010}], où la modélisation du rôle des erreurs d'un seul élément dans les lentilles à rayons X est utilisée dans le ray-tracing, l'analyse étant cependant limitée à la largeur et à l'intensité du faisceau dans le plan image. La modélisation d'erreurs de forme plus complexes a été rendue possible par l'utilisation des polynômes orthonormaux de Zernike ou de Legendre 2D. Cela a été fait en unifiant certaines parties déjà existantes des bibliothèques Python, en ajoutant de nouveaux ensembles de polynômes et en les interfaçant pour les rendre compatibles avec le cadre déjà décrit. Cette bibliothèque Python unifiée et étendue a été utilisée tout au long de la thèse pour ajuster les erreurs de forme présentées tout au long des chapitres~\ref{sec:modelling}-\ref{sec:corrections}. La modélisation des erreurs de forme réalisée soit en accordant plus de degrés de liberté au modèle de lentille idéal, soit en générant directement les erreurs de forme par des polynômes orthonormaux, est un outil utile pour définir les tolérances de fabrication des lentilles rayons X, ce qui est particulièrement pertinent pour le projet de fabrication en interne de lentilles en aluminium. Toute cartographie arbitraire des imperfections de surface (par exemple issue de données de métrologie ou de surfaces free-form pour la mise en forme du faisceau) peut également être intégrée dans les simulations. Plusieurs auteurs ont développé de nombreux outils avec différents degrés de complexité pour simuler les lentilles rayons X et leurs aberrations. Toutefois, non seulement ces outils ne sont pas accessibles au public, mais la modélisation n'est souvent pas compatible avec les outils de simulation déjà largement répandus pour la conception des optiques rayons X, tels que SHADOW (ray-tracing) et SRW (propagation du front d'onde). La modélisation présentée ici est de type open source, pouvant être utilisée de manière transparente avec SRW et est actuellement disponible dans un dépôt public sur GitLab jusqu'à la fusion éventuelle avec la distribution officielle de SRW. Parallèlement, le code est incorporé dans la boîte à outils OASYS [\cite{Rebuffi2017}] qui sera disponible pour les distributions de SHADOW, du ray-tracing hybride et de SRW. Ces développements récents sur la modélisation des lentilles rayons X ont été présentés dans [\cite{Celestre2020b}] et utilisés dans certaines des simulations de [\cite{Chubar2020}].

Finalement, pour réaliser les simulations avec des niveaux d'imperfections réalistes, la modélisation des erreurs de forme sur les lentilles rayons X individuelles a été faite avec des données de métrologie sous forme de cartographie des déviations locales du profil parabolique. La technique de métrologie utilisée pour cette thèse était le XSVT (X-ray vectorial speckle tracking). En plus d'être relativement simple à mettre en œuvre, cette technique possède une résolution latérale élevée et une bonne sensibilité. De plus, elle fournit une cartographie des erreurs de forme dans l'approximation de la projection, qui peut être facilement utilisée avec les simulations de front d'ondes. 
Plusieurs campagnes de mesures ont été effectuées à l'ESRF, à l'APS et à l'ESRF-EBS pour évaluer la qualité des lentilles produites en interne, des lentilles et des optiques free-form dans le cadre de collaborations scientifiques et des lentilles commerciales nouvellement acquises. En conséquence, une base de données importante et diversifiée de fichiers de métrologie a été constituée. Une expansion possible de DABAM (base de données pour la métrologie des miroirs rayons X) [\cite{SanchezDelRio2016}] est envisagée. La mise à disposition des données de métrologie des miroirs rayons X par le biais d'une base de données open-source aiderait à consolider le projet développé ici puisque DABAM est également distribuée et interfacée par OASYS. Auparavant, il faudra cependant parvenir à la convergence des résultats de metrologie obtenus sur un empilement de lentilles avec celui du calcul logiciel réalisé à partir de la métrologie des lentilles qui le composent. L'état actuel montre un certain accord qualitatif (Fig.~\ref{fig:accumulated_profile_1}-\ref{fig:CDo})), qui est transféré aux simulations présentées dans les Figs.~\ref{fig:CDn_vs_CDnStack}-\ref{fig:CDo_vs_CDoStack} et Fig.~\ref{fig:CD_Strehl}. Cela est suffisant pour les études préliminaires mais laisse une marge de manœuvre pour améliorer et délimiter les protocoles de métrologie, tant en termes d'alignement des échantillons que de traitement des données - ce qui devrait être l'objectif de futurs études. L’intérêt majeur de pouvoir empiler de manière fiable des lentilles mesurées individuellement étant de pouvoir évaluer les performances d'une CRL composée d’une sélection arbitraire de lentilles, et concevoir les correcteurs optiques nécessaires. L'expérience acquise lors des campagnes de métrologie et les progrès réalisés dans le traitement des données acquises ont permis d'apporter de modestes contributions à [\cite{Berujon2020a,Berujon2020,Qiao2020b}].

Les simulations, présentées dans la première partie du chapitre ~\ref{sec:effect_optical_imperfections} et résumées par la figure ~\ref{fig:CDnS},, montrent une caustique du faisceau qui a été mentionnée plusieurs fois dans la littérature pour le même type de lentilles rayons X [\cite{Schropp2013,Seiboth2017,Seiboth2020}]. La mesure expérimentale de la caustique du faisceau pour cet empilement de lentilles montre également une similarité avec ce qui a été prédit par les simulations - cf. Fig.~\ref{fig:experimental_CDn_pp}. L’accord entre simulations et données expérimentales contribue à valider l’approche du sujet de cette thèse et on estime que les simulations permettent d'évaluer qualitativement les effets des imperfections optiques sur la dégradation du profil du faisceau de rayons X. Elles permettent également d'étudier la pertinence du rapport de Strehl comme facteur de qualité pour la conception de systèmes optiques fonctionnant loin du critère de Maréchal pour la qualité optique. En général, les équations utilisées pour estimer le rapport de Strehl tendent à sous-estimer la performance du système dans cette région (Fig.~\ref{fig:Strehl}). L’impossibilité d'obtenir des équations de remplacement simples (cf. Eq.~\ref{eq:Strhel_Exp}) indique que la manière la plus simple d'évaluer le rapport de Strehl est la simulation optique du système en utilisant des erreurs de forme réalistes. Il faut également souligner que bien que les simulations entièrement et partiellement cohérentes ne diffèrent pas de manière significative, il est prématuré de généraliser ce comportement car les simulations partiellement cohérentes sont basées sur une hypothétique ligne de lumière fonctionnant avec le réseau magnétique ESRF amélioré. Un autre résultat des simulations est la prédiction que les systèmes avec des erreurs de forme à haute fréquence spatiale ne présentent pas un élargissement de la taille du faisceau au plan focal, mais que l'intensité est plutôt réduite par la diffusion des photons autour du lobe principal, ce qui augmente le bruit de fond - Fig.~\ref{fig:hf_strehl_scan}. Les effets des erreurs de forme à haute fréquence spatiale sont importants, car pour des systèmes optiques bien corrigés ils seraient une source prédominante d'erreurs de forme. Cette analyse a été publiée dans [\cite{Celestre2020}].


La modélisation multi-coupes d’une CRL incluant les données métrologiques des lentilles mesurées individuellement (Eq.~\ref{eq:TE_CRL_MS_ERR}) permet l'extraction des erreurs de forme cumulées d'un empilement de lentilles. Ceci a été utilisé pour modéliser une plaque de phase en diamant destinée à corriger ces erreurs en sortie du système. Afin de faciliter l’alignement de cette plaque, celle-ci est conçue de façon symétrique au détriment de la performance corrective - les résultats rapportés montrent une excellente performance malgré ce compromis [\cite{Seiboth2020,Dhamgaye2020}]. Le profil du faisceau illustré par la figure~\ref{fig:CDn_corrected} est plus homogène en amont et en aval du plan image et ressemble aux simulations pour un système à prédominance de hautes fréquences spatiales - figure~\ref{fig:CDnHF}. Cependant, lors de l'analyse du rapport de Strehl, la performance de la plaque de phase modélisée est limitée et dépasse ce que d'autres groupes ont rapporté. Les principales raisons de ces divergences sont le fait que le profil à corriger (Fig.~\ref{fig:accumulated_profile_1}) comporte des composantes non symétriques importantes (aberration en trèfle) en plus de la prédominance classique des aberrations sphériques. L'absence d'asymétrie constatée par d'autres groupes est probablement liée au fait qu'un plus grand nombre de lentilles sont utilisées dans leurs empilements, ainsi les erreurs corrélées des lentilles dont la position en rotation est aléatoire seraient rendues plus symétriques par rapport aux erreurs accumulées par un effet de moyennage des composantes non symétriques. La rotation aléatoire des lentilles dans les simulations pourrait contribuer à améliorer le rapport de Strehl attendu et une procédure similaire pourrait être appliquée à l’empilement de lentilles dans le cadre d'une expérience. Les plaques de phase conçues ont été commandées à un partenaire commercial et sont obtenues par ablation du diamant avec un laser femtoseconde. Les premiers essais sur un faisceau de rayons X ont montré que le centrage des plaques de phase dans leur cadre ne respectait pas les tolérances calculées, ce qui est crucial pour leur performance. En l’absence de protocole d’alignement lors des premiers essais, cela a posé un problème, la plaque de phase devant être alignée par rapport à la pile de lentilles. Cela a également mis en évidence la nécessité de réimplémenter des techniques de détection de front d'onde plus rapides, qui ont été abandonnées au profit d'une résolution spatiale plus élevée. D'autres problèmes tels que la surexposition du détecteur, même après une forte atténuation du faisceau, ont limité les tests qui pouvaient être effectués avec la plaque corrective, de sorte qu'aucune valeur quantitative n'a pu être obtenue pour évaluer la performance de la correction. Cependant, une caustique du faisceau a pu être enregistrée pour le système aberré et corrigé (Fig.~\ref{fig:experimental_CDn_pp}) montrant qualitativement un faisceau plus homogène surtout au voisinage du plan focal, avec une meilleure performance en aval. Les premiers résultats sont prometteurs mais nécessitent de programmer de nouvelles expériences sur ligne de lumière, de trouver des moyens d'extraire rapidement le front d'onde résiduel et d'aligner la plaque de correction, de mesurer la caustique du faisceau et d'évaluer quantitativement les performances de la correction. L'exploration de la fabrication additive pour la conception d'optiques correctives dans d'autres matériaux devrait également être étudiée comme alternative au diamant et pour réduire les coûts de prototypage. $\blacksquare$

\addcontentsline{toc}{section}{Références}
\printbibliography[heading=subbibliography]
\end{refsection}