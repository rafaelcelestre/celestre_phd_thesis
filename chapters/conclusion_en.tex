\begin{refsection}\renewcommand{\thechapter}{7.en}
\chapter{~~Conclusion}\label{sec:conclusion_en}

The newly debuted 4$^\text{th}$ generation of storage rings puts stringent requirements on the quality of optical elements as to minimise the degradation of the X-ray beam [\cite{Schroer2014,Yabashi2014}]. The aim of this thesis, investigating and modelling the effect of optical imperfections in compound refractive lenses on a partially-coherent X-ray beam, addresses important aspects of the ESRF-EBS X-ray optics R$\&$D programme as laid out in the Orange book [\cite{orangebook}]. 

Based on the physical optics concepts presented in Chapter~\ref{sec:x-rays} and the already implemented optical elements in SRW [\cite{Baltser2011}], an expanded model for a thick ideal CRL was presented. This model is similar to the multi-slicing technique already used for optical modelling [\cite{Li2017,Ali2020}], where each lens is considered to be a slice of the CRL. The complex transmission element representing a thick CRL is given by Eq.~\ref{eq:TE_CRL_MS} and shown in Fig.~\ref{fig:models}(b). This ideal model of a thick CRL can be modified to account for optical imperfections when a 2D map of the figure errors of each lens is available. This model is shown in Fig.~\ref{fig:models}(c) and described by Eq.~\ref{eq:TE_CRL_MS_ERR}. This modelling of CRL as a thick optical element becomes more relevant when the lens stack is composed of a large number of lenslets and there is significant focusing inside the CRL [\cite{Schroer2005}]. The extended modelling of CRLs was published in [\cite{Celestre2020}].

Once the framework for the modelling thick CRLs was defined, the attention was funnelled to modelling individual lenses. Typical misalignments and fabrication errors commonly present in embossed X-ray lenses (Fig.~\ref{fig:lens_cuts}) were parametrised by allowing more degrees of freedom to the ideal model for a single lens described in [\cite{Baltser2011}]. This parametrisation is general enough to apply to lenses produced by other techniques, but this modelling could profit from adding the choice of the conic section to be used, currently limited to the parabolic case, as already implemented by [\cite{SanchezdelRio2012,Andrejczuk2010}] for ray-tracing. Some groups have already experimented with non-parabolic shapes for focusing X-rays [\cite{Alianelli2007,Evans-Lutterodt2003, Alianelli2015, Sutter2017}] and the availability of simulation tools could revive the interest in non-orthodox designs. It would be amiss not to mention the similarities between the modelling presented here and the work presented by [\cite{Andrejczuk2010}], where the modelling of the role of single element errors in X-ray lenses is implemented for ray-tracing, but the analysis is limited to the beam width and intensity in the image plane. The modelling of more intricate shape errors was enabled by employing the Zernike or 2D Legendre orthonormal polynomials. This was done by unifying some already existing parts of Python libraries, adding new polynomial sets and interfacing them to be compatible with the framework already described. This unified and expanded Python library was used throughout the thesis for fitting the figure errors presented throughout Chapters~\ref{sec:modelling}-\ref{sec:corrections}. The modelling of shape errors by either allowing more degrees of freedom to the ideal lens model or by directly generating the shape errors through orthonormal polynomials is a useful tool for delineating tolerances on the manufacturing of X-ray lenses, which is of particular relevance for the in-house fabricated Al-lenses project. Any arbitrary 2D map of surface imperfections (eg. metrology data or free-form surfaces for beam shaping) can also be incorporated in the simulations using this framework. Several authors have developed numerous tools with different degrees of complexity for simulating X-ray lenses and their aberrations. However, not only are those tools not available to the public but also the modelling is often not compatible with already existing wide-spread simulation tools for X-ray optical design such as SHADOW (ray-tracing) and SRW (wavefront propagation). The modelling presented here is open source, tailored to be used transparently with SRW and is currently available in a public repository in GitLab until the eventual merge with the SRW official distribution. Parallel to it, the code is being incorporated in the OASYS toolkit [\cite{Rebuffi2017}] to be made available for their distributions of SHADOW, hybrid ray-tracing and SRW. These recent developments on X-ray lenses modelling were presented in [\cite{Celestre2020b}] and used in some of the simulations from [\cite{Chubar2020}].

Ultimately, for simulations with realistic levels of imperfections, the modelling of the figure errors on individual X-ray lenses was done with metrology data as 2D maps of local deviations of the parabolic profile. The metrology technique used for this thesis was the X-ray vectorial speckle tracking (XSVT). In addition to being relatively simple to implement, this technique has a high lateral resolution and good sensitivity. Furthermore, it delivers a 2D map of figure errors in projection approximation, which can conveniently be used with the wave-optical simulations. 
Several measurement campaigns at the ESRF, APS and ESRF-EBS were performed to assess the quality of in-house produced lenses; lenses and free-form optics in the context of scientific collaborations; and newly-acquired commercial lenses. As a result, a large and diverse database of metrology files was built. After some curation, a possible expansion of the DABAM (database for X-ray mirrors metrology) [\cite{SanchezDelRio2016}] is envisioned. Making the metrology data of X-ray lenses available through an open-source database would help the consolidation of the framework developed here since DABAM is also distributed and interfaced by OASYS. Before that, however, more effort has to be put into making the metrology of single lens and subsequent staking by software converge with the metrology of a lens stack composed by those same lenslets. The current state show some qualitative agreement (Fig.~\ref{fig:accumulated_profile_1}-\ref{fig:CDo}), which is transferred to the simulations shown in Figs.~\ref{fig:CDn_vs_CDnStack}-\ref{fig:CDo_vs_CDoStack} and Fig.~\ref{fig:CD_Strehl}. This is sufficient for preliminary studies but leaves room for improvements and delineation of protocols for the metrology both in terms of sample alignment and data processing - these should be the aim of future work. The major importance of reliably being able to artificially stack individually measured lenses is the possibility of evaluating the performance of a CRL composed of any arbitrary selection of lenses off-line, with immediate application to the design of optical correctors. The experience gained in the metrology campaigns and advancements in the post-processing of the acquired data resulted in modest contributions to [\cite{Berujon2020a,Berujon2020,Qiao2020b}].

A subset of lenses in the database was chosen to showcase the effect of optical imperfections on partially coherent X-ray beam degradation. 
The simulations, shown in the first part of Chapter~\ref{sec:effect_optical_imperfections} and summarised by Fig.~\ref{fig:CDnS}, show a beam caustic that has been reported several times in literature for the same type of X-ray lenses [\cite{Schropp2013,Seiboth2017,Seiboth2020}]. The experimental measurement of the beam caustics for this lens stack also shows similarity to what was predicted by the simulations - cf. Fig.~\ref{fig:experimental_CDn_pp}. The similarity between the simulations and experimental data helps to validate the framework presented in this thesis and it is believed that they allow to qualitatively assess the effects of optical imperfections on the degradation of the X-ray beam profile. The simulations also allow investigating the adequacy of the Strehl ratio as a figure of merit for the optical design of systems operating far from the Mar\'echal criterion for optical quality. Generally, the equations used for estimating the Strehl ratio tend to underestimate the system performance in that region (Fig.~\ref{fig:Strehl}). The failed attempt in obtaining simple replacement equations (cf. Eq.~\ref{eq:Strhel_Exp}) indicate that the most straightforward way to evaluate the Strehl ratio is by optical simulation of the system using realistic figure errors. It is also relevant to mention that although fully- and partially coherent simulations do not differ significantly, it is premature to generalise this behaviour as the partially-coherent simulations were done for a hypothetical undulator beamline operating at the upgraded ESRF magnetic lattice. Another result from the simulations is the prediction that the systems with high spatial frequency figure errors do not present a broadening of the beam size at the focal plane, instead, the intensity is reduced by the scattering of photons around the main lobe, increasing the background level - Fig.~\ref{fig:hf_strehl_scan}. The effects of high-spatial-frequency figure errors are important as well-corrected optical systems would have them as a predominant source of shape errors. This analysis was published in [\cite{Celestre2020}].

The multi-slice modelling of a CRL including the metrology data of individually measured lenses (Eq.~\ref{eq:TE_CRL_MS_ERR}) allows the extraction of the accumulated figure errors of a lens stack. This was used to model a phase-plate in diamond to correct for those errors at the exit pupil of the system. Due to the limitations of the phase-plate alignment in a beamline, the symmetric design is preferred in detriment to the correction performance - reported results show excellent performance even with this trade-off [\cite{Seiboth2020,Dhamgaye2020}]. The beam profile shown in Fig.~\ref{fig:CDn_corrected} is more homogeneous up- and downstream the image plane and resembles the simulations for a system with predominantly high-spatial frequencies in Fig.~\ref{fig:CDnHF}. However, when analysing the Strehl ratio, the performance of the modelled phase-plate is limited and beyond what other groups have reported. The main reasons for such discrepancies are the fact that the profile to be corrected (Fig.~\ref{fig:accumulated_profile_1}) has significant non-symmetric components (trefoil aberration) in addition to the classical predominance of spherical aberrations. The lack of asymmetry found by other groups is likely connected to the fact that a larger number of lenses are used in their stacks and the addition of correlated errors of randomly rotated lenses would act to give a more symmetric shape to the accumulated errors by averaging out the non-symmetric components. Randomly rotating the lenses in the simulations could help to improve the expected Strehl ratio and similar procedure could be applied to the lens stack in an experiment. The designed phase-plates were commissioned from a commercial partner and they were ablated from diamond by a femtosecond laser. Initial tests on an X-ray beam showed that the phase plates were not centred in the frame within the calculated tolerances, which is crucial for their performance. The phase-plate had to be aligned concerning the lens stack, which posed an initial problem, as at the time no alignment protocols were delineated. This also brought to the surface the necessity of re-implementation of faster wavefront sensing techniques which were discontinued in favour of a higher spatial resolution. Other issues such as overexposure of the detector even after strong attenuation of the beam limited the tests that could be done with the correction plate, so no quantitative value was generated to evaluate the correction performance, however, a beam caustic could be recorded for the aberrated and corrected system (Fig.~\ref{fig:experimental_CDn_pp}) qualitatively showing a more homogeneous beam especially in the vicinity of the focal plane, with a better performance downstream. 

The early results are promising but show the necessity of re-planning the next beamtimes and looking for alternative ways for rapidly extracting the residual wavefront and aligning the correction plate, measuring the beam-caustics and quantitatively evaluating the correction performance. Exploring additive manufacturing for designing the corrective optics in other materials should also be investigated as an alternative to diamond and lowering the prototyping costs. $\blacksquare$

\addcontentsline{toc}{section}{References}
\printbibliography[heading=subbibliography]
\end{refsection}

% The necessary code modifications  to model such experimentally determined phase errors also open the avenue to calculating the effects of any arbitrary phase modifying element.