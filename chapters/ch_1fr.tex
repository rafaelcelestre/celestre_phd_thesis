% \appendix

\chapter*{Préface}\addcontentsline{toc}{chapter}{Préface}

L'amélioration de la qualité des sources de rayons X modernes impose des exigences de plus en plus strictes à la qualité de l'optique des rayons X. Les optiques à rayons X utilisées pour focaliser les faisceaux de rayons X ne doivent pas dégrader la brillance des sources de rayons X. Pour y parvenir, la perturbation du front d'onde des rayons X, les effets négatifs sur les points focaux et les pertes d'intensité doivent tous être minimisés. Pour les LCR, cela nécessite une fidélité de la forme des lentilles, des surfaces lisses et une structure interne homogène et pure. Ce projet de doctorat vise à mesurer les propriétés de l'optique des rayons X et à les comparer aux simulations informatiques qui prennent en considération la taille de la source et le degré de cohérence attendu pour l'anneau de stockage EBS amélioré. Il comporte donc à la fois une partie expérimentale et une partie de simulation. Il permettra de mesurer l'effet sur le faisceau de rayons X de chaque lentille individuelle par le biais de l'imagerie en champ proche basée sur le speckle ou de techniques comparables, de prédire l'effet d'une pile de LCR sur le faisceau et de vérifier les propriétés de focalisation de la pile avec le faisceau de rayons X. Dans un premier temps, cela ne sera possible qu'avec le faisceau ESRF actuel moins cohérent, mais des essais dans le faisceau plus cohérent de l'EBS devraient être possibles au cours de la troisième année. Ce projet aborde des aspects importants du programme R$\&$D d'optique des rayons X de l'EBS, tel que décrit dans le \textit{Orange book}.

\section*{Motivation}\addcontentsline{toc}{section}{Motivation}

This PhD project aims at determining the effect of real, imperfect X-ray optics on a partially coherent X-ray beam, characteristic of the ESRF source after the EBS upgrade. In particular, the effect of refractive lens shape errors, surface roughness, and impurities on a partially coherent X-ray beam shall be simulated. In order to obtain realistic results, experimental speckle-tracking wavefront sensing techniques shall be used to characterise X-ray lenses and other optics. Novel lenses currently under development in the X-ray optics group are 2D focusing kinoform lenses in SU-8 polymers, 1 and 2D diamond as well as in-house fabricated Al-lenses. 3D printing and other fabrication techniques will be investigated to determine their suitability for use in the manufacture of corrective optics for imperfect lens or mirror systems.

\section*{Objectives}\addcontentsline{toc}{section}{Objectives}

\begin{itemize}
\item Wave-optics simulations: simulations using available tools (SRW, COMSYL, Shadow, ...) on partially coherent beams passing through 2D focusing X-ray lenses featuring realistic levels of imperfections: a) shape errors, b) surface roughness, c) inclusions and voids. The influence of optical imperfections shall be investigated for a partially coherent beam and the possibility to use corrective optics components to mitigate some of the wavefront degradation will be explored;

\item Study the limitations of the thin object approximation used in most simulation codes for dealing with the optical element imperfections. Evaluate its impact on the simulations for the imperfection in CRL. If necessary, develop methods to go beyond this approximation, for example by solving the Maxwell equations for a surface with imperfections illuminated by an X-ray wavefront via Finite Element Methods;
	
\item Speckle based imaging for wavefront sensing measurements on real lenses and arrays with the aim to correct shape errors with corrective optics produced by various fabrication methods (e.g. 3D printing, FIB machining, etc).
	
\end{itemize}

\section*{Aperçu}\addcontentsline{toc}{section}{Aperçu}

This work is divided into six chapters, discussion and a conclusion. The relevant publications during this PhD project are summarised in an appendix at the end of this work. This PhD thesis is organised as follows:
\\
\\
\textbf{Chapter \ref{sec:x-rays}~-~\nameref{sec:x-rays}} is the first of two theoretical chapters, presents ...
\\
\\
\textbf{Chapter \ref{sec:x-ray_optics}~-~\nameref{sec:x-ray_optics}}: introduces ... this chapter concludes the presentation of the theoretical aspects necessary for this dissertation.
\\
\\
\textbf{Chapter \ref{sec:modelling}~-~\nameref{sec:modelling}} introduces ...
\\
\\
\textbf{Chapter \ref{sec:measuring}~-~\nameref{sec:measuring}} introduces ...
\\
\\
\textbf{Chapter \ref{sec:effect_optical_imperfections}~-~\nameref{sec:effect_optical_imperfections}} introduces ...
\\
\\
\textbf{Chapter \ref{sec:corrections}~-~\nameref{sec:corrections}} introduces ...
\\
\\
\textbf{Chapter \ref{sec:conclusion_en}~-~\nameref{sec:conclusion_en}}  ...
\\
\\
\textbf{Appendix \ref{sec:publications}~-~\nameref{sec:publications}}  ...

$\blacksquare$