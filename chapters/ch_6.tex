\begin{refsection}
\chapter{Correcting optical imperfections in refractive lenses}\label{sec:corrections}

The necessity and possibility of arbitrarily manipulating the X-ray wavefront has been teased since, at least, the early 2000's [\cite{Chubar1999, Chubar2001b}]. It was not until the advent of extremely accurate additive and subtractive manufacturing techniques [\cite{Stohr2015, Polikarpov2016, Petrov2017, Roth2018, Sanli2018, Seiboth2019, Antipov2020, Medvedskaya2020}] that the demonstration of free-form X-ray refractive optics was done [\cite{Seiboth2017,Seiboth2020}]. The possibility of producing very accurate free-form optics for optical aberrations has brought renewed interest in wavefront sensing and optical design simulations. This chapter presents the early results on correcting optical imperfections in refractive lenses obtained at the ESRF. The design and expected performance is based on the lenses presented in \S\ref{sec:single_lens}~-~\textit{\nameref{sec:single_lens}} and the simulations shown in Chapter~\ref{sec:effect_optical_imperfections}-~\textit{\nameref{sec:effect_optical_imperfections}}.


%-------------------------------------------------------------------------
%-------------------------------------------------------------------------
\section{Corrective optics}
%-------------------------------------------------------------------------
%-------------------------------------------------------------------------
\section{Design}\label{sec:design}

\section{Prototype}\label{sec:prototype}

\section{Expected performance}\label{sec:performance}

\section{Early tests on an X-ray beam}\label{sec:prototype_testing}


% %-------------------------------------------------------------------------


$\blacksquare$
\addcontentsline{toc}{section}{References}
\printbibliography[heading=subbibliography]
\end{refsection}
