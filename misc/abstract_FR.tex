% 13/11/2020 - First round of corrections: added Thomas' corrections
\pdfbookmark[0]{Résumé}{Résumé}
\chapter*{Résumé}
\label{sec:abstractFR}
\vspace*{-10mm}

L'avènement des installations synchrotron à haute énergie de quatrième génération (ESRF-EBS et les projets APS-U, PETRA-IV et SPring-8 II) et des lasers à électrons libres (Eu-XFEL, SLAC), allié à la récente démonstration d'optiques réfractives à forme libre de haute qualité pour la mise en forme des faisceaux et la correction optique, a ravivé l'intérêt pour les lentilles réfractives composées (LRC) en tant qu'optiques pour le transport des faisceaux, la formation de sondes dans la micro- et la nano-analyse des rayons X ainsi que pour les applications d'imagerie. Dans ce contexte, en 2018, l'ESRF a repris la fabrication et les essais de lentilles biconcaves à rayons X en aluminium à focalisation 2D. Les optiques réfractives actuelles, commerciales ou autres, présentent des aberrations non négligeables qui détériorent leurs performances finales et une modélisation précise avec l'aide de la métrologie est nécessaire pour évaluer la dégradation du front d'onde et proposer des stratégies d'atténuation.

En optique physique, les éléments faiblement focalisants sont généralement simulés comme un seul élément mince dans l'approximation de projection. Alors qu'une seule lentille à rayons X dans des conditions de fonctionnement typiques peut souvent être représentée de cette manière, la simulation d'un LRC complet avec une approche similaire conduit à un modèle idéalisé qui manque de polyvalence. Ce travail propose de décomposer un LRC en ses petites lentilles séparées par une propagation en espace libre, ce qui ressemble aux techniques de découpage multiple (MS) déjà utilisées pour les simulations optiques. L'attention est portée sur la modélisation de l'élément à lentille unique en lui ajoutant des degrés de liberté supplémentaires permettant la modélisation des désalignements typiques et des erreurs de fabrication. Des polynômes orthonormaux pour les aberrations optiques ainsi que des données de métrologie en rayons X sont également utilisés pour obtenir des résultats de simulation réalistes, qui sont présentés dans plusieurs simulations cohérentes et partiellement cohérentes tout au long de ce travail. Ils se comparent qualitativement bien avec les données expérimentales et sont utilisés pour évaluer l'effet des imperfections optiques sur la dégradation du faisceau de rayons X partiellement cohérent et la pertinence des figures de mérite communes. Contrairement à d'autres travaux, la modélisation présentée ici peut être utilisée de manière transparente avec l'un des codes les plus populaires pour la conception de lignes de faisceaux de rayons X, "Synchrotron Radiation Workshop" (SRW), et est disponible sur GitLab.

Les imperfections optiques mesurées avec une haute résolution spatiale peuvent être ajoutées à la représentation MS d'un LRC pour représenter avec précision de véritables lentilles à rayons X. Le suivi vectoriel du speckle des rayons X (XSVT) est une technique polyvalente qui permet d'obtenir facilement les erreurs de figure des lentilles de rayons X dans l'approximation de projection avec une haute résolution spatiale et est utilisée dans ce travail pour caractériser les lentilles 2D-beryllium qui sont ensuite utilisées dans les modélisations présentées ici. Cette thèse présente une revue des concepts les plus pertinents de l'imagerie basée sur le speckle des rayons X appliquée à la métrologie des lentilles.

La mise en œuvre du modèle MS d'un LRC en utilisant des données de métrologie permet d'extraire les erreurs cumulées des lentilles empilées et permet le calcul des corrections de phase. Cette thèse se termine par la présentation d'une méthodologie de calcul du profil des correcteurs de réfraction, qui est appliquée pour produire des plaques de phase ablatées au diamant. Les premiers résultats expérimentaux montrent une amélioration du profil du faisceau, mais l'alignement transversal de la plaque est un facteur limitant. D'autres améliorations de la métrologie des lentilles et des plaques de correction, des efforts dans la délimitation des protocoles d'alignement sont nécessaires pour amener la conception des correcteurs optiques dans l'ESRF.