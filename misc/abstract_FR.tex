% 13/11/2020 - First round of corrections: added Thomas' corrections
\pdfbookmark[0]{Résumé}{Résumé}
\chapter*{Résumé}
\label{sec:abstractFR}
\vspace*{-10mm}

Le déploiement des installations synchrotron à haute énergie de 4e génération (ESRF-EBS et les projets APS-U, HEPS, PETRA-IV et SPring-8 II) et des lasers à électrons libres (Eu-XFEL, SLAC) allié aux récents développements d’optiques réfractives « free-form » de haute qualité visant à conditionner le faisceau de rayons X, a ravivé l’intérêt pour les lentilles réfractives composées (CRL) permettant la propagation du faisceau, ou son conditionnement pour la micro et la nano-analyse, ou encore pour les applications d’imagerie. Dans ce contexte, l'ESRF a repris en 2018 la fabrication et les tests de lentilles bi-concaves en aluminium à focalisation 2D. Les optiques réfractives actuelles, commerciales ou non, présentent des aberrations qui détériorent leurs performances finales. Aussi, une modélisation précise incluant des données de métrologie est nécessaire pour évaluer la dégradation du front d'onde afin de proposer des stratégies d’amélioration. 

En optique physique, les éléments faiblement focalisant sont généralement simulés comme un seul élément mince dans l'approximation de projection. Alors qu'une seule lentille rayons X dans des conditions de fonctionnement typique peut souvent être représentée de cette manière, la simulation d'une CRL entière avec une approche similaire conduit à un modèle idéalisé qui manque de polyvalence. Ce travail propose de décomposer une CRL en ses lentilles élémentaires séparées par une propagation en espace libre, comme dans la technique dite de multi-coupes (multi-slicing - MS) déjà utilisées dans les simulations optiques. L'attention est portée sur la modélisation de la lentille élémentaire en lui ajoutant des degrés de liberté supplémentaires permettant de simuler des désalignements typiques ou des erreurs de fabrication. Des polynômes orthonormaux décrivant les aberrations optiques ainsi que des données de métrologie obtenues avec les rayons X sont également utilisés pour obtenir des résultats de simulation réalistes, qui sont présentés dans plusieurs simulations cohérentes et partiellement cohérentes tout au long de ce travail. Les résultats ainsi obtenus se comparant qualitativement bien avec les données expérimentales, sont utilisés pour évaluer l'effet des imperfections optiques sur la dégradation du faisceau de rayons X partiellement cohérent ainsi que la pertinence de facteurs de mérite communs. Contrairement à d'autres travaux, la modélisation présentée ici peut être utilisée de façon transparente avec l'un des codes les plus populaires pour la conception de lignes de faisceaux de rayons X, "Synchrotron Radiation Workshop" (SRW), et est disponible sur GitLab.

Les imperfections optiques mesurées avec une haute résolution spatiale peuvent être ajoutées à la représentation MS d'une CRL pour simuler avec précision de vraies lentilles rayons X. Le suivi vectoriel du speckle des rayons X (XSVT) est une technique polyvalente qui permet d'obtenir facilement les erreurs de forme des lentilles rayons X dans l'approximation de projection avec une haute résolution spatiale. Elle a été utilisée pour caractériser les lentilles 2D-beryllium qui sont ensuite utilisées dans les modélisations présentées ici. Cette thèse présente une revue des concepts les plus pertinents de l'imagerie basée sur le speckle des rayons X appliquée à la métrologie des lentilles.

La mise en œuvre du modèle MS d’une CRL incluant des données de métrologie permet d’extraire les erreurs cumulées résultantes de l’empilement des lentilles ainsi que le calcul des corrections de phase. Cette thèse se termine par la présentation d'une méthodologie de calcul du profil des correcteurs de réfraction, qui est appliquée pour produire des plaques de phase ablatées en diamant. Les premiers résultats expérimentaux montrent une amélioration du profil du faisceau, mais l'alignement transversal de la plaque est un facteur limitant. Des améliorations concernant la métrologie des lentilles et des plaques de correction, ainsi que les protocoles d'alignement seront nécessaires optimiser les performances de ces correcteurs optiques.
