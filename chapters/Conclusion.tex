\begin{refsection}
\chapter{Conclusion}
\label{sec:conclusion}

Building on physical optics concepts and already implemented optical elements in SRW, we have expanded the concept of the complex transmission element representation of the CRL to account not only for its thick element nature but also real imperfections obtained with at-wavelength metrology. We have studied the adequacy of commonly used design equations and figures of merit by doing coherent and partially-coherent simulations. We were able to accurately simulate the effects of figure errors on beam shape and intensity along the optical axis. Our simulations of the beam caustics compare well with experimental data from other research groups using the same type of Be lenses. We show that using the Strehl ratio formulations given by Eqs.~\ref{eq:Strehl}-\ref{eq:Mahajan} leads to an underestimation of the system performance if the total projected figure errors are larger than the limit imposed by the $\lambda/14$ criterion (Mar\'echal criterion for optical quality). We see an immediate application to lens tolerancing and guidelines for accepting or not commercial optical elements and in-house lens production and testing of X-ray lenses (quality control). By decomposing the figure errors in frequency ranges, we note that the strongest contribution to the degradation of the wavefront both in focus and away from it comes from the low-frequency range, which is where corrective optics are most efficient. By being able to add individual lens profiles to a lens stack we envisage the possibility of calculating corrective optics for an arbitrary lens combination offline, as opposed to experimentally measuring the wavefront phase errors of the lens stack as proposed by [\cite{Seiboth2017}]. 

[\cite{SanchezdelRio2011}] and hybrid methods [\cite{Shi2014}]

$\blacksquare$
\addcontentsline{toc}{section}{References}
\printbibliography[heading=subbibliography]
\end{refsection}