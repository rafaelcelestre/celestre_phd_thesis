\chapter*{Préface}\addcontentsline{toc}{chapter}{Prelude}


Improvements in the quality of modern X-ray sources place increasingly stringent demands upon the quality of X-ray optics, which when used to focus X-ray beams should not degrade the X-ray source brilliance. To achieve this, the perturbation of the X-ray wavefront, adverse effects on the focal spots, and intensity losses all need to be minimised. To achieve that with refractive optics, X-ray lenses shape fidelity, smooth surfaces and a homogeneous pure internal structure. 

This PhD project aimed at determining the effect of refractive lens shape errors, surface roughness, and impurities on a partially coherent X-ray beam with similar characteristics of the undulator beamlines after the ESRF-EBS upgrade. Based on recent developments, the mitigation of lens shape error employing corrective optics was also investigated. To achieve the proposed goals, this project was based on two pillars: theoretical and experimental with technical aspects related to both. This project addressed important aspects of the ESRF-EBS X-ray optics R$\&$D programme as laid out in the strategic upgrade plane (Orange book).

The theoretical facet of the work involved studying the limitations of current modelling and approximations used in most simulation codes for dealing with the optical elements. After evaluating the adequacy of the existing tools, the proposition of extensions and new developments. Adding to the simulations the capability of handling metrology data and developing a framework for the design of refractive corrective optics was also aimed. Integrate the simulation models to coherent- and partially-coherent simulations to realistically obtain the effect of optical imperfections on an X-ray beam and compare the results with the literature and experimental data. The technical goals related to this theoretical part included the development of Python libraries for easily using the newly developed modelling with "Synchrotron Radiation Workshop" (SRW) code for beamline design. Part of this implementation was developed in collaboration with O. Chubar (SRW author) during two scientific visits to the Brookhaven National Lab. in the U.S.A. This newly-developed Python library was the base for further integration in user graphical interfaces like OrAnge SYnchrotron Suit (OASYS). 

To obtain realistic results for the simulations, near-field X-ray speckle-based wavefront sensing techniques were used to characterise in-house produced lenses; lenses and free-form optics in the context of scientific collaborations; and newly-acquired commercial lenses. Beamtimes at-wavelength metrology were routinely conducted at the BM05 at the ESRF before its shutdown in early December 2018, 1-BM beamline at the APS in Chicago during the year of 2019 and at the ID06 beamline, during the ESRF-EBS commissioning period. Besides the measurement of X-ray optics and curation of a metrology database for X-ray lenses, the technical goals of this experimental part were training the PhD candidate to be able to perform with understanding and proficiency and autonomy: the setting up the experimental setup at the beamline, data acquisition and data processing. Developing internal measurement protocols in terms of probe alignment and standardising the measurements and data analysis were also expected. 

At a later stage, investigating recent developments in additive and subtractive manufacturing techniques for the manufacturing of optical correction was also required. Designing and testing the performance on an X-ray beam of the first phase correctors at the ESRF was envisioned as a natural way of concluding the PhD training.

\section*{Outline}\addcontentsline{toc}{section}{Outline}

This work is divided into six chapters, a conclusion and one appendix summarizing the author's relevant publications during this PhD project:
\\
\\
\textbf{Chapter \ref{sec:x-rays}~-~\nameref{sec:x-rays}} is the first of two predominantly theoretical chapters. The title is a nod to the title of the publication reporting the discovery of X-rays in 1895. It begins by introducing the concepts of brilliance and connecting it to accelerator-based X-ray sources. High-brilliance X-ray sources are then presented and the concept of laterally coherent X-rays are shown. Physical optics is then presented as the most appropriate description of (partially-) coherent fields. Free-space propagation and the paraxial approximation are explained and the propagation of X-ray though matter is modelled through the introduction of the transmission element. A short discussion on optical coherence the presentation of some basic concepts is done at the end of this section. This chapter finishes with a discussion on X-ray optical simulations, methods and approaches. Throughout the text, extensive recommended literature is presented.
\\
\\
\textbf{Chapter \ref{sec:x-ray_optics}~-~\nameref{sec:x-ray_optics}}, a reference to A. Compton's Nobel lecture, opens with a historical recount of the early days of X-ray science showing the milestones that led to the understanding of X-ray as a branch of optics. A short review of the developments in X-ray focusing optics divided by optical phenomena is given for contextualising how recent refractive optics is. The ideal X-ray lens modelling and the ideal lens stack based on multi-slicing-like techniques are presented. With little modification, this model can accept arbitrary 2D figure error maps to account for optical imperfections. Important metrics for evaluating the CRL performance are introduced and tolerance conditions for aberrations are presented. This chapter concludes the presentation of the theoretical aspects necessary for this dissertation.
\\
\\
\textbf{Chapter \ref{sec:modelling}~-~\nameref{sec:modelling}} focuses on modelling the X-ray lenslet. By adding to the front- and back- focusing surfaces lateral- and angular- degrees of freedom, it is possible to mimic typical misalignments and fabrication errors encountered in real lenses. For the cases where the newly parametrised degrees of freedom are not enough, the modelling of more intricate shape errors is enabled by employing the Zernike or 2D Legendre orthonormal polynomials or metrology data. This chapter finishes with the details of the Python libraries implemented for modelling phase imperfections in X-ray lenses.
\\
\\
\textbf{Chapter \ref{sec:measuring}~-~\nameref{sec:measuring}} presents a complete description of X-ray near field speckle vector tracking (XSVT) technique used to inspect the lenses used in this thesis. It begins by describing the plurality of at-wavelength metrology techniques and delineates why XSVT is the most appropriate technique to be used in this work. A review of the main aspects of the experimental setup,  data acquisition, processing and analysis is presented and a discussion on the metrology of X-ray lenses vs. lens stacks closes this chapter.
\\
\\
\textbf{Chapter \ref{sec:effect_optical_imperfections}~-~\nameref{sec:effect_optical_imperfections}} discusses the effect of optical imperfections on the X-ray beam degradation by presenting an extensive collection of simulations of beam-caustics, the point-spread function and the beam profile at selected positions along the optical axis for several optical setups. A discussion on the metrology of individual lenses vs. stacked lenses from the point of view of the simulations are presented, followed by a discussion on the effect of optical imperfections and the adequacy of the Strehl ratio for X-ray lenses. Some comments on the simulations time conclude the chapter.
\\
\\
\textbf{Chapter \ref{sec:corrections}~-~\nameref{sec:corrections}} is the last chapter and concludes the journey that began in \textbf{Chapter~\ref{sec:x-ray_optics}} with the modelling of ideal lenses, passing through the modelling of aberrated lenses in \textbf{Chapter~\ref{sec:modelling}}, measuring the very same imperfections in \textbf{Chapter~\ref{sec:measuring}} and understanding the their effects on an X-ray beam in \textbf{Chapter~\ref{sec:effect_optical_imperfections}}. This chapter begins by listing important milestones in extremely accurate additive and subtractive manufacturing techniques that enabled producing very accurate free-form optics for the correction of optical aberrations. A review on strategies for correcting optical imperfections in X-ray optics is given and a methodical method for the refractive correction phase plate calculation is given. Using the simulations tools developed for this thesis, the expected performance of the correction plate can be calculated and alignment tolerances can be drawn. A prototype in diamond is shown and ear test on an X-ray beam are shown to demonstrate a qualitative improvement on the beam profile. A lengthy discussion on the design and expected performance versus the early phase plate tests on an X-ray beam is presented at the end of this chapter.
\\
\\
\textbf{Chapter \ref{sec:conclusion_en}~-\nameref{sec:conclusion_en}} summarising the significance, implications, contributions and limitations of this thesis and laying out future directions.
\\
\\
\textbf{Appendix \ref{sec:publications}~-~\nameref{sec:publications}} of the works produced by the PhD candidate during this project.