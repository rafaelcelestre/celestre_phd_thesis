\pdfbookmark[0]{Abstract}{Abstract}
\chapter*{Abstract}
\label{sec:abstractEN}
\vspace*{-10mm}

The advent of 4$^th$ generation high-energy synchrotron facilities (ESRF-EBS and the planned APS-U, PETRA-IV and SPring-8 II) and free-electron lasers (Eu-XFEL, SLAC) allied with the recent demonstration of high-quality free-form refractive optics for beam shaping and optical correction has revived interest in compound refractive lenses (CRLs) as optics for beam transport, probe formation in X-ray micro- and nano-analysis as well as for imaging applications. 

Within this context, in 2018, the ESRF resumed the fabrication and tests of 2D focusing bi-concave Aluminium X-ray lenses. Available refractive optics have non-negligible aberrations which degrade their final performance. accurate modelling modelling necessary to study preserve wavefront and the beam degradation from real lenses.

This thesis presents a set a framework based on physcial optics and teh extensoin of MS technique for ... ... need metrology.

A review on speckle is provided ... wavefront sensing... measurement of X-ray lenses.

With the complete model, the simulations are done and are able to reprocide experimental data

calcualtion of phase paltes


% which will be required to preserve wavefront and withstanding the more brilliant photon beam


% which required the  accurate modelling and metrology techniques (wavefront sensing).

% Within the wave-optics framework, weakly focusing elements are usually simulated as a single thin element in the projection approximation. While a single X-ray lens at typical operation conditions can often be represented in this way, simulating a full CRL with a similar approach leads to an idealised model that lacks versatility. Decomposing a CRL into its individual lenslets separated by a free-space propagation is an approach that resembles multi-slice techniques (MS).

% Optical imperfections measured with high spatial resolution can be readily converted into a transmission element and can be added to the MS representation of a CRL. At-wavelength metrology is often the most appropriate for obtaining such figure errors. Each error profile used in the modelling presented here comes from 2D-Be lenses individually characterised using X-ray speckle vectorial tracking (XSVT) at the BM05 - Instrumentation Beamline at the ESRF. 

% With coherent and partially coherent simulations using the SRW, we present the effects of misalignments, manufacturing errors and other phase errors obtained from at-wavelength metrology of X-ray lenses. Applications to tolerancing and manufacturing of corrective optics, as well as a framework for those will be presented.

% The advent of 4th generation high-energy synchrotron facilities (ESRF-EBS and the planned APS-U, PETRA-IV and SPring-8 II) and free-electron lasers (Eu-XFEL, SLAC) allied with the recent demonstration of high-quality free-form refractive optics for beam shaping and optical correction has revived interest in compound refractive lenses (CRLs) as optics for beam transport, probe formation in X-ray micro- and nano-analysis as well as for imaging applications. Ideal CRLs have long been made available in the 'Synchrotron Radiation Workshop' (SRW), one of the most popular codes for X-ray beamline design, however, the current context requires more sophisticated modelling of such X-ray lenses. In this work, we revisit the already implemented wave-optics model for an ideal X-ray lens in the projection approximation and propose modifications to it as to allow more degrees of freedom to both the front and back surfaces independently, which enables to reproduce misalignments and manufacturing errors commonly found in pressed lenses (most abundant type - eg. Be and Al lenses). We present the effects of each new degree of freedom by simulating their impact on the point spread function, the beam caustics and on the phase immediately after the back surface of the lens. The improved X-ray lens model remains compatible with the multi-slicing (MS) modelling of a CRL. A simulation of a lens stack using the MS model of a CRL is also presented here. We conclude by showing some applications of accurate X-ray lens modelling to manufacturing and at-wavelength metrology applications at the ESRF.


% Compound refractive lenses (CRL) were developed in the mid-1990s at the European Synchrotron Radiation Facility (ESRF) as a simple method for focusing X-rays [1]. CRL offer outstanding potential for generating small probe beams in X-ray micro- and nano-analysis as well as for imaging applications, but effective aberration-free focusing of X-rays is a key consideration when high spatial resolution and intensities are required. Available refractive optics have non-negligible aberrations which degrade their final performance. We describe a framework based on physical-optics for simulating the effect of imperfect CRL upon an X-ray beam. Coherent and partially-coherent simulations using the "Synchrotron Radiation Workshop" (SRW) [2] are used to evaluate the different models.
% Within the wave-optics framework, weakly focusing elements are usually simulated as a single thin element in the projection approximation. While a single X-ray lens at typical operation conditions [3] can often be represented in this way, simulating a full CRL with a similar approach leads to an idealised model that lacks versatility and – for example – overestimates absorption. Decomposing a CRL into its individual lenslets separated by a free-space propagation is an approach that resembles multi-slice techniques (MS). The MS modelling of stacks of individual lenses allows for simulating typical lens misalignments and manufacturing errors (cf. Fig. 1).

% Optical imperfections measured with high spatial resolution can be readily converted into a transmission element and can be added to the MS representation of a CRL as shown in Fig. 2. At-wavelength metrology is often the most appropriate for obtaining the figure errors. Each error profile used in the modelling presented here comes from 2D-Be lenses individually characterised using X-ray speckle vectorial tracking (XSVT) [4] at the BM05 - Instrumentation Beamline at the ESRF. The introduction of at-wavelength metrology data to the modelling phase imperfections in compound refractive lenses was first introduced in [5].
% With coherent and partially coherent simulations using the SRW, we present the effects of misalignments, manufacturing errors and other phase errors obtained from at-wavelength metrology of X-ray lenses. Applications to tolerancing and manufacturing of corrective optics, as well as a framework for those will be presented. 

% Acknowledgements
% The authors are thankful to O. Chubar (NSLS-II/BNL) for the helpful discussions, to C. Detlefs (ID06/ESRF) for discussions and providing the lenses for metrology and to R. Cojucaro (BM05/ESRF) for help during beamtimes.

% References
% [1]	Snigirev, A. et al, Nature 384, 49–51 (1996)
% [2]	Chubar, O. and Elleaume, P., Proc .EPAC-98, 1177-1179 (1998)
% [3]	Roth, T. et al, MRS Bulletin 42(6),430-436 (2017)
% [4]	Berujon, S. et al, arXiv:1902.09418 & arXiv:1912.05432 (2019)
% [5]	Celestre, R. et al, arXiv:1911.06712 (2019)
